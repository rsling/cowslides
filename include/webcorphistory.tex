
\subsection{Erste und zweite Generation}

\begin{frame}
	{Erste Generation: WaCky und frühe WAC Workshops}
	\begin{itemize}
  		\item ursprüngliches Ziel: BNC-Ersatz umsonst
  		\item keine echte Methodendiskussion aus linguistischer Sicht
  		\item WaCky-Korpora: keine Metadaten, destruktiv normalisiert,\\linguistische Nachverarbeitung minimal, nicht angepasst
  		\item COW11 und COW12: ähnlich wie WaCky 
  		
  		\vspace{0.5cm}
  			
  		\item als wir anfingen: bereits deutliche Erscheinungen\\von Diversifikation von SIGWAC und Entwicklung zu\\einer Kommerzialisierung von Webkorpora 
	\end{itemize}
\end{frame}

\begin{frame}
	{Zweite Generation?}
	\begin{itemize}
	  \item DeRiK: \alert{Minikorpora}, handselegiert nach Art des DWDS-KK\\\alert{viel zu klein} für Graphematik, Morphologie, Syntax
	  \item SketchEngine\slash Lexical Computing: technologisch hervorragend,\\ausgerichtet auf Lexikographie, \alert{teuer}
	  \item Glowbe von BYU: fragwürdige Art der Erstellung und Verbreitung, außerdem \alert{teuer}
	\end{itemize}
\end{frame}

\begin{frame}
	{Was macht COW/COCO anders?}
	\begin{itemize}
	  \item technologisch auf dem höchsten Stand\\(Bereinigung, Normalisierung, Annotation, Metadaten)
	  \item keine Optimierung auf lexikographische Fragestellungen
	  \item sehr \alert{groß}: Variation im Nichtstandard und\\niederfrequente Phänomene untersuchbar
	  \item \alert{frei} soweit möglich (s.\ unten zu Rechtslage und COCO)
	  
	  \vspace{0.5cm}
	   
	  \item \alert{COW ist das einzige verbleibende Webkorpusprojekt\\mit eigener Technologie auf Höhe der Zeit,\\das große freie Korpora baut!}
	\end{itemize}
\end{frame}

\subsection{Rechtslage}

\begin{frame}
	{Das deutsche Urherberrecht}
	\begin{itemize}
	  \item umfassender Schutz des Werkes vor jeglicher Verwendung,\\der der Autor nicht zugestimmt hat
	  \item anders als viele denken: Art der Verwendung\\(kommerziell vs.\ Wissenschaft) spielt \alert{keine Rolle}
	  \item für jeden Text explizite Erlaubnis erforderlich
	  \item verwaiste Werke nicht automatisch frei, erheblicher Aufwand beim Auffinden des Urhebers muss nachgewiesen werden
	  \item gemacht für Bücher und Opern -- pervers für Blogs, Foren\\oder von Spiegel Online mit Rechtschreibfehlern versehene DPA-Meldungen  
	\end{itemize}
\end{frame}

\begin{frame}
	{Lösung}
	\begin{itemize}
		\item langfristig: Ausnahmeregelungen für Wissenschaft
		\item Fair Use gilt als inkompatibel zu deutschem Urheberrecht
		\item also \alert{vollständige Reform}, einschließlich Verwertungsgesellschaften 
		\item ohne Lobbyarbeit und Allianzen mit anderen Disziplinen\\keine Chance für Korpuslinguitik
		\item Initiative aus GSCL AK DigHum und ACL SIGWAC:\\erste Podiumsduskussion bei WAC-X @ ACL 2016\\12.\ August in Berlin
		\item Alternative: Unterwanderung, s. COCO
	\end{itemize}
\end{frame}

