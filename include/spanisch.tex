%\documentclass{beamer}
%\usepackage{etex}
%\usepackage[utf8]{inputenc}
%\usepackage[OT1]{fontenc}
%\usepackage[ngerman]{babel}
%%\usepackage[utf8x]{inputenc}
%\usepackage{colortbl}
%\usepackage{uzk}
%\usepackage{natbib}
%\usepackage{my-gb4e-slides}
%%\usepackage{felix-gb4e-slides}
%\usepackage{verbatim}
%\usepackage{url}
%\usepackage{bbding}
%\usepackage{amssymb}
%\usepackage{amsmath}
%
%\newcommand{\done}{\cellcolor{teal}done}  %{0.9}
%\newcommand{\hcyan}[1]{{\color{teal} #1}}
%
%%\definecolor{beamer@fuorange}{rgb}{.9,0.5,0.1}
%\definecolor{beamer@fuorange}{rgb}{.8,.3,0}
%\definecolor{beamer@fudarkgreen}{rgb}{0,0.4,0}
%\definecolor{Gray}{gray}{0.5}
%\definecolor{Black}{gray}{0}
%\setbeamercolor{alerted text}{fg=beamer@fuorange}
%
%\let\citew=\citealp
%
%\newcommand{\cites}[1]{\citeauthor{#1}'s \citeyearpar{#1}}
\newcommand{\bluecell}{\cellcolor[rgb]{.7,.7,.9}}
\newcommand{\bluerow}{\rowcolor[rgb]{.7,.7,.9}}
\newcommand{\bluecol}{\columncolor[rgb]{.7,.7,.9}}
%\newcounter{lastpagemainpart}
%\setcounter{tocdepth}{1}
%
%\newcommand{\xxx}{\hspaceThis{[}}
%\newcommand{\Lf}{
%  \setlength{\itemsep}{1pt}
%  \setlength{\parskip}{0pt}
%  \setlength{\parsep}{0pt}
%}
%\newcommand{\graw}[1]{{\color[rgb]{0.5,0.5,0.5}#1}}
%\newcommand{\blaw}[1]{{\color[rgb]{0.2,0.2,0.9}#1}}
%\newcommand{\grien}[1]{{\color[rgb]{0.1,0.6,0.1}#1}}
%\newcommand{\myalert}[2]{{\color<#1>[rgb]{0,0.7,0}#2}}
%\newcommand{\eg}{e.\,g.}
%\newcommand{\Eg}{E.\,g.}
%\newcommand{\ie}{i.\,e.}
%\newcommand{\Ie}{I.\,e.}
%\newcommand{\Dim}{\cellcolor{Gray}}
%\newcommand{\Off}{\cellcolor{Black}}
%
%
%
%\begin{document}

% Includes hier! --------------------------------

\subsection{VSO in Spanish}

%\begin{frame}
%Fallstudie: VSO-Strukturen im Spanischen\\
%
%\footnotesize{gefördert von SFB 632/Informationsstruktur} 	
%\end{frame}

\begin{frame}
  \frametitle{VSO in Spanish}

Neutrale, unmarkierte Konstituentenabfolge im Spanischen: \textbf{SVO}\\[.3ex]
{\footnotesize (Contreras 1976; Hernanz \& Brucart 1987;
  Meirama 1997; Delbecque 2005; Gutierrez 2006, u.\,v.\,a.)}

\pause

\begin{itemize}
  \item[$\rightarrow$] Aber: je nach Verb/Verbklasse auch \textbf{VS} und \textbf{OVS}
\end{itemize}

\vspace{.5cm}

\pause

Andere Abfolgen sind nicht neutral/unmarkiert, sondern informationsstrukturell eingeschränkt

\begin{itemize}
  \item[$\rightarrow$] Oft beschrieben: \textbf{VOS} und \textbf{SV}
\end{itemize}

\vspace{.5cm}
 
\pause

Weniger oft beschrieben und immer noch ein wenig rätselhaft: \textbf{VSO}




\end{frame}


\begin{frame}
  \frametitle{VSO}


\begin{exe}
  \ex{[Buscaron]$_V$ [los padres]$_S$ [a la niña]$_O$ llenos de dolor sin encontrarla por ningún lado.}\\
    {\footnotesize `Voller Schmerz suchten die Eltern das Mädchen überall, ohne sie zu finden.'}

  \ex{[Deja]$_V$ [Di Stéfano]$_S$ [la dirección técnica]$_O$ y llega Amancio para dirigir al equipo.}\\
  {\footnotesize `Di Stéfano verlässt den Trainerposten und es kommt Amancio um die Mannschaft zu führen.'}
  
  \ex{[Posee]$_V$ [Huelva]$_S$ [unas increíbles playas de arena blanca]$_O$ \ldots}\  
      {\footnotesize `Huelva besitzt einige unglaubliche Strände mit weißem Sand'}
\end{exe}
\end{frame}


\begin{frame}
  \frametitle{VSO}\begin{itemize}
%\item relativ selten% 7--15\% deklarativer Hauptsätze in gesprochener
%  Sprache (berichtet in \citealp{Suner1982})
\item abhängig von der Textsorte (Gonzales de Serralde 2001)
\item abhängig von der regionalen Varietät\\ (z.\,B.\ nicht im mexikanischen
  Spanisch, Gutierrez 2006)
\item tritt \textit{nicht} in eng verwandten Sprachen auf, z.\,B.\ Katalanisch (Leonetti 2014), Italienisch (Belletti 2001)% im  frühen 19.\ Jhd.\ aber noch möglich, cf.\ \citealp{Wandruszka1982})
\end{itemize}
  
\end{frame}



\begin{frame}
  \frametitle{Eigenschaften von VSO}
  Einige intuitive/spekulative Beschreibungen, wenig empirisch fundiertes.\\[1ex]
  \centering
  \scalebox{.5}{  
\begin{tabular}{|p{3cm}|p{2cm}|p{2cm}|p{2cm}|p{2cm}|p{2cm}|}
\hline
                               & Suner (1982)      & Neumann-Holzschuh (1997) &  Zubizarreta (1998)  & Ordonez (2000) & Leonetti (2014)\\
\hline
Satzfokus                      &   \grien{\Checkmark}  & \alert{\XSolidBrush}        & \grien{\Checkmark}       & \grien{\Checkmark} & \grien{\Checkmark}\\
\hline
Subjekt im Fokus               &   \grien{\Checkmark}  
                                   (kontrastiv)        & \alert{\XSolidBrush}        &\grien{\Checkmark}
                                                                                       (kontrastiv)
                                                                                                                &\grien{\Checkmark} & \alert{\XSolidBrush}\\
\hline
Subjekt + Objekt im Fokus      &                       & \alert{\XSolidBrush}        &                          & \grien{\Checkmark} & \grien{\Checkmark}\\
\hline
\bluerow eine (einzige) Informationseinheit  &                  &   \alert{\XSolidBrush}      &                          &                   & \grien{\Checkmark}\\
\hline
\bluerow mehrere Informationseinheiten  &                       &  \grien{\Checkmark}         &                          &                   &\alert{\XSolidBrush}\\
\hline
Reihung von Ereignissen        &   \grien{\Checkmark}  &                             &                          &                  &  \\
\hline
Subjekt immer Teil
 der ``Assertion''             &                       &                             &                          & \grien{\Checkmark} & \grien{\Checkmark}\\
\hline
Subjekt gegeben                &                       &  \grien{\Checkmark}         &                          &                   &  \\
\hline
\end{tabular}
}
 
\end{frame}
\begin{frame}
\frametitle{Informationsstruktureller Status}

Es gibt fundamental verschiedene Annahmen über den informationsstrukturellen Status von \textbf{VSO}.\\[.3cm]


Neumann-Holzschuh (1997), empirische Studie:\\
\begin{itemize}
%\item ``markierte Variante'' kategorischer (rumnänisch)
\item kategorische Äußerung (Thema/Rhema-Gliederung)
\item Subjekt ist diskursgegeben und Thema
%\item beantwort die Frage: ``Und was geschah dann mit der gennannten Person/Sache?''
\item ``narrative Konstruktion'': zeigt \textit{topic-continuity} an
\end{itemize}

\pause
\vspace{.2cm}

Leonetti (2014), ``theoretische'' Überlegungen\\
(als ``Korpusstudie'' getarnt): 

 \begin{itemize}
  \item nur thetische Interpretation, keine informationsstrukturelle Gliederung
 \end{itemize}

\end{frame}


\begin{frame}
\frametitle{Frage}

Ist VSO eine Strategie zur ``De-Topikalisierung''?

    \begin{itemize}
    \item Das Subjekt wird aus der kanonischen Topikposition herausgenommen.
    \item Entsteht dabei eine Struktur, die als wide-focus/thetisch/event-reporting zu interpretieren ist?
    \end{itemize}
	
\end{frame}


\begin{frame}
 \frametitle{VSO: informationsstrukturell gegliedert oder nicht?}

Definition: ``thetisch'' 
\begin{quote}
   A thetic statement is uttered only if the elements that would constitute the predication base and the predicate in a corresponding categorical statement both convey (textually) new information.
\end{quote} (Sasse 1987: 567)

\pause
\vspace{.2cm}

Wenn VSO-Sätze thetisch sind, dann sind hier weniger diskursgegebene Subjekte zu erwarten, als bei SVO.

\end{frame}



\begin{frame}
  \frametitle{Korpusstudie (ESCOW12; 1,6 GT)}
  
  \begin{itemize}
    \item $N=20,000$ Verb-initiale Sätze   
    \item manuell VSO Strukturen herausfiltern:\\
      \begin{itemize}
      \item ``O'' ist direktes oder indirektes Objekt
      \item ``O'' ist eine NP oder PP (aber keine infinite VP oder Objektsatz)
      \end{itemize}
\end{itemize}
\pause

``Gesäuberte'' Stichprobe: 78 Belege ($0.4\%$) für VSO

\begin{itemize}
    \item Annotation von Diskursgegebenheit von Subjekt und Objekt\\
          (RefLex-Schema, Riester et al. 2010)
  \end{itemize}
\end{frame}

\begin{frame}
  \frametitle{Vergleich mit SVO}

$\rightarrow$ ``entsprechende'' Stichprobe von SVO-Sätzen.  

  \begin{itemize}
    \item dieselben Verb-Lemmata wie in der VSO-Stichprobe, in den gleichen Anteilen
    \item nur deklarative Hauptsätze
    \item möglichst kein Adverbial vor SVO (also nicht XP-SVO) 
    \item entsprechend für ``Diskursgegebenheit'' annotieren
  \end{itemize}
\end{frame}


\begin{frame}
  \frametitle{Vergleich mit SVO (2)}

Wenn VSO-Sätze thetisch sind, dann sind hier weniger diskursgegebene Subjekte zu erwarten, als bei SVO.

\pause

\vspace{.3cm}

\begin{figure}
	\begin{tabular}{lrr}
	\hline
	         & SVO & VSO\\
\hline
   R-bridge  & 20  & 14\\
   R-given   & 45 & 53 \\
   R-new     & 13 & 11\\
	\hline	
	\end{tabular}
\end{figure}

\vspace{.3cm}

Kein statistisch signifikanter Unterschied (aber die Tendenz geht sogar in die nicht-erwartete Richtung).

\end{frame}



\begin{frame}
  \frametitle{VSO: typischerweise Topik-Kommentar Struktur}

  \begin{itemize}
  %\item Das Subjekt muss nicht Teil des Fokus sein.% (contra \citealp{Ordonez2000}).
 \item Die große Mehrzahl der Subjekte in VSO sind referentiell gegeben oder inferierbar $(85\%)$.%; some are \textit{unused-known}
  \item Die Mehrzahl der Fälle sind wahrscheinlich Topik-Kommentar-Strukturen.
  \end{itemize}
\pause

\begin{exe}
  \ex
  \begin{xlist}
    \ex  El poco interés que suscitan lo demuestra \blaw{el esquema de tratamiento habitual}$_i$.%\\

  \ex Consiste \alert{este}$_i$ \grien{en una referencia breve al comienzo del estudio de los movimientos} \ldots\\
   %        consists this in a reference brief to.the beginning of.the study of the movements\\
           {\footnotesize `The latter consists in briefly mentioning the beginning of the study of movements'} 
\end{xlist}
\end{exe}  

\end{frame}


\begin{frame}
  \frametitle{VSO: typischerweise Topik-Kommentar Struktur (2a)}

  \begin{itemize}
  \item Neumann-Holzschuh (1997): VSO als abgeschwächte Topik-Kommentar-Strukturen
  \item Subjekte sind \textit{continuation topics}
  \item bevorzugt in narrativen Texten Gonzales de Serralde (2001)
  \end{itemize}
\end{frame}



\begin{frame}
  \frametitle{VSO: typischerweise Topik-Kommentar Struktur (2b)}
\scalebox{0.8}{\begin{minipage}{\textwidth}
  \begin{exe}
    \ex
    \begin{xlist}
      \ex \ldots nos contaron la leyenda del levantamiento del cruceiro manifestando que , cuando el lugar de Cerqueiras era todo un robledal , había muchos lobos en él y cierto día se llevaron los lobos a una niña pequeña, hija de \blaw{Xan de Porto y Catalina de Ribeira}$_i$ , que figuran en la inscripción del cruceiro .

\ex Buscaron \alert{los padres}$_i$ \grien{a la niña} llenos de dolor sin encontrarla por ningún lado.

\ex \blaw{$\emptyset$}$_i$ Acudieron entonces a la Virgen que en la capilla recibe homenaje y \blaw{$\emptyset$$_i$} le pidieron fervorosamente intercesión en el caso. Cuando \blaw{$\emptyset$$_i$} volvieron a buscar, \blaw{$\emptyset$$_i$} dieron con la niña viva y sana acariciando un lobo muerto que a su lado tenía. \blaw{Los padres}$_i$ agradecidos hicieron levantar el cruceiro que allí está.


    \end{xlist}

  \end{exe}
  \end{minipage}}
\end{frame}


 \begin{frame}
  \frametitle{Zusammenfassung}

VSO ist kompatibel mit verschiedenen informationsstrukturellen Partitionierungen
  \begin{itemize}
%  \item fokussiertes Subjekt (eher selten)
%  \item wide-focus/event-reporting/thetic (eher selten)
  \item[$\rightarrow$] charakteristisch: ``abgeschwächte'' Topik-Kommentar Struktur
 \end{itemize}

\pause
\vspace{.5cm}


\alert{VSO-Sätze haben typischerweise eine informationsstrukturelle Gliederung.}

\vspace{.5cm}

Also ist VSO typischerweise keine Strategie der De-Topikalisierung.

%\vspace{.5cm}
%
%\begin{itemize}
%  \item Man muss (viele) Belege in ihrem Kontext analysieren.
%  \item Man darf Möglichkeiten nicht a priori (aus ``theoretischen'' Gründen) ausschließen.
%  \item Selbst in Webkorpora (fast ohne narrative Texte) finden sich Daten für eine entsprechende Analyse.
%\end{itemize}




 \end{frame}


% What is VSO, then?


% \begin{itemize}
% \item so far, no specific information structural motivation
% \item probably a stylistic variant
% \item apparently correlated with certain text types and/or genres
% \item attracts specific pronominal subjects (\textit{usted})
% \item restricted in its discourse function, but rather not on grounds of information structure
% \end{itemize}




% Includes hier zuende! -------------------------
%
% \begin{frame}[allowframebreaks]
% {References}
%\def\newblock{\hskip .11em plus .33em minus .07em}
%\tiny
%\bibliographystyle{natbib.fullname} 
%\bibliography{BigBiblio}
%\end{frame}
%
%\makeatletter
%\setcounter{lastpagemainpart}{\the\c@framenumber}
%\makeatother
%%\appendix
%%\section{Anhang}
%
%%\include{appendix_roland}
%
%\mode<beamer>{\setcounter{framenumber}{\thelastpagemainpart}}
%\end{document}
